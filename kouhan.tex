\documentclass{jsarticle}
\usepackage{amsthm}
\newtheorem{thm}{定理}
\newtheorem{fom}[thm]{公式}

\begin{document}

\title{離散構造(後半)レポート}
\author{16B04852 川原和弥}
\maketitle

このレポートでは、授業中の配布資料における定理\ref{ringoku}を証明する。

\begin{thm}[隣国は5つだけ定理]
    \label{ringoku}
    $どんな地図にも、5個以下の隣国しか持たない国が少なくとも一つは存在する。$
\end{thm}

\begin{proof}
    背理法により証明する。\\
    \ \ \ すなわち、すべての国が6個以上の隣国を持つような地図があるとする。\\
    \ \ \ この地図の$(面,辺,頂点)$の数を$(F,E,V)$とおくと、
    \begin{eqnarray}
    1つの頂点に集まる辺は3本以上 \Rightarrow  E \geq \frac{3}{2} V \\
    1つの面の境界になる辺は6本以上 \Rightarrow  E \geq \frac{6}{2} F
    \end{eqnarray}

    また任意の地図上で次の公式\ref{Euler}が成り立つ。\\

    
    \begin{fom}[Eulerの多面体公式]
        \label{Euler}
        \begin{center}$\ F - E + V = 2$\end{center}
    \end{fom}


    (1),(2)より、
    \begin{eqnarray}
    F - E + V &\leq& \frac{2}{6} E - E + \frac{2}{3} E \nonumber \\
    &=& 0 \nonumber
    \end{eqnarray}

    しかしこれは公式\ref{Euler}に反する。

    これにより、すべての国が6個以上の隣国を持つ地図が存在しないことが示された。

\end{proof}

お詫び:第3回目以降の授業に出席しておらずレポートの内容がわかりませんでしたので、資料を参考に本レポートを作成しました。
もし見当違いな内容でしたら申し訳ありません。

\end{document}
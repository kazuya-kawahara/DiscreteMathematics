\documentclass{jsarticle}
\usepackage{amsthm}
\newtheorem{lem}{補題}
\newtheorem{fom}[lem]{公式}

\begin{document}


\title{離散構造(後半)レポート}
\author{16B04852 川原和弥}
\date{}
\maketitle


\begin{enumerate}


\item
\begin{enumerate}
\renewcommand{\labelenumii}{(\arabic{enumii})}
\item $ F - E + V = 0 - 3 + 3  = 0 $
\item $ F - E + V = 1 - 3 + 3  = 1 $
\item $ F - E + V = 0 - 5 + 4  = -1 $
\item $ F - E + V = 1 - 5 + 4  = 0 $
\item $ F - E + V = 1 - 7 + 5  = -1 $
\end{enumerate}


\item
\begin{proof}

まず次の補題\ref{ringoku}を示す。

\begin{lem}[隣国は5つだけ定理]
\label{ringoku}
\begin{eqnarray} どんな地図にも、5個以下の隣国しか持たない国が少なくとも一つは存在する。 \nonumber \end{eqnarray}
\end{lem}

補題\ref{ringoku}の証明. 背理法により証明する。\\
すなわち、すべての国が6個以上の隣国を持つような地図があるとする。\\
この地図の$(面,辺,頂点)$の数を$(F,E,V)$とおくと、
\begin{eqnarray}
1つの頂点に集まる辺は3本以上 \Rightarrow  E \geq \frac{3}{2} V \\
1つの面の境界になる辺は6本以上 \Rightarrow  E \geq \frac{6}{2} F
\end{eqnarray}
また任意の地図上で次の公式\ref{Euler}が成り立つ。
\begin{fom}[Eulerの多面体公式]
\label{Euler}
\begin{eqnarray} F - E + V = 2 \end{eqnarray}
\end{fom}
(1),(2)より、
\begin{eqnarray}
F - E + V \leq \frac{2}{6} E - E + \frac{2}{3} E = 0 \nonumber
\end{eqnarray}
しかしこれは(3)に反する。
これにより、すべての国が6個以上の隣国を持つ地図が存在しないことが示された。
\qed \\

以上の補題\ref{ringoku}を踏まえ、問題である$5$色定理を示す。

地図上の国の数$N$に関する帰納法により証明する。
\begin{enumerate}
\item $ N = 1 $のとき\\
任意の色で塗ればよく、自明である。
\item $ N = k $で5色定理が成り立つとする。\\
$ N($M$) = k + 1 $である地図Mを$5$色で塗ることを考える。
補題\ref{ringoku}によりMには隣国が$5$個以下の国が存在するから、その国をCとする。
MからCを除いた地図をM$'$とすると、$ N($M$') = k + 1 $だから、帰納法の仮定によりM$'$は$5$色で塗り分けられる。
\begin{enumerate}
\item Cが5個の隣国を持ち、全て異なる色で塗られているとき
\begin{enumerate}
\item
\item

\end{enumerate}
\item そのほかの場合\\
Cの隣国に使われていない色でCを塗れば良い。

\end{enumerate}

\end{enumerate}

\end{proof}

\item

\end{enumerate}


\end{document}